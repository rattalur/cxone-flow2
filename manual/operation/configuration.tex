\chapter{Configuration}


\section{Runtime Configuration}\label{sec:runtime-config}

\subsection{SSL}

\subsubsection{Trusting Self-Signed Certificates}\label{sec:self-signed-certs}

While the \cxone system uses TLS certificates signed by a public CA, it is possible that
corporate proxies use certificates signed by a private CA. Private CA certificates must be imported
into the \cxoneflow container.

Each private CA certificate for import must meet the following criteria:

\begin{itemize}
    \item It must be in a file ending with the extension .crt.
    \item The contents of the file must be one certificate stored in the PEM format.
    \item All files containing private CA certificates must be mapped to the container path \texttt{/usr/local/share/ca-certificates}.
\end{itemize}


As an example, if using Docker, it is possible to map a single local file to a file in the container with this mapping 
option added to the container execution command line:

\begin{code}{Custom CA Mapping Option}{[Docker]}{}
-v $(pwd)/custom-ca.pem:/usr/local/share/ca-certificates/custom-ca.crt
\end{code}

\subsubsection{The \texttt{ssl-verify} Option}\label{sec:ssl-verify-general}

In the configuration YAML documentation, all of the \texttt{connection}
elements contain an optional \texttt{ssl-verify} setting.  This option
is generally useful to turn off SSL verification by setting it to \texttt{False}.
This can also be used to control which CA bundle is used for verification.

Omitting the \texttt{ssl-verify} setting should be sufficient for
most deployment cases.  If omitted, the container execution will use the default CA bundle
where any custom CAs are added as described in Section \ref{sec:self-signed-certs}.
The \texttt{ssl-verify} option can be set to an explicit path on the container
if there is a need to use a CA bundle other than the one provided by the OS.


\subsubsection{Configuring SSL for the \cxoneflowtext Endpoint}

Configuring the \cxoneflow endpoint for SSL communication requires an SSL certificate public/private key pair
and map the files to a location on the container.  The following environment variables must then be set in the
runtime environment:

\begin{table}[ht]
    \caption{SSL Environment Variables}
    \begin{tabularx}{\textwidth}{ll}
        \toprule
        \textbf{Variable} & \textbf{Description}\\
        \midrule
        \texttt{SSL\_CERT\_PATH} & \makecell[l]{The path to the server's SSL certificate in PEM format.}\\
        \midrule
        \texttt{SSL\_CERT\_KEY\_PATH} & \makecell[l]{The path to the certificate's unencrypted private key in PEM format.}\\
        \bottomrule
    \end{tabularx}
\end{table}

If your SSL certificate is self-signed, the certificate must also be imported as the CA as described
in Section \ref{sec:self-signed-certs}.  If the certificate is signed with a private CA, the private
CA must also be imported.  Failure to import a non-public signing CA for these types of certificates
will cause \cxoneflow startup failures.


\subsection{Runtime Control Environment Variables}

Environment variables can be set when the \cxoneflow container is executing to control some aspects of \cxoneflowns's operation.
Table \ref{tab:runtime-environment-vars} shows the operational environment variables and their meaning.

\begin{table}[ht]
    \caption{Runtime Control Environment Variables}\label{tab:runtime-environment-vars}
    \begin{tabularx}{\textwidth}{lccl}
        \toprule
        \textbf{Variable} & \textbf{Required} & \textbf{Default} & \textbf{Description}\\
        \midrule
        \texttt{CXONEFLOW\_WORKERS} & No & \texttt{max(\# of CPUs / 2, 1)} & \makecell[l]{The number of worker processes\\used for parallel execution. The\\maximum value will be\\set at \texttt{(\# of CPUs - 1)}}\\
        \midrule
        \texttt{LOG\_LEVEL} & No & \texttt{INFO} & \makecell[l]{The logging verbosity level.  Set to\\\texttt{DEBUG} for increased logging\\verbosity.}\\
        \midrule
        \texttt{CONFIG\_YAML\_PATH} & No & \texttt{/opt/cxone/config.yaml} & \makecell[l]{The path to the configuration\\YAML file.}\\
        \midrule
        \texttt{CXONEFLOW\_HOSTNAME} & No & \texttt{localhost} & \makecell[l]{The virtual hostname of the\\\cxoneflow endpoint.}\\
        \bottomrule
    \end{tabularx}
\end{table}


\newpage

\section{Operational Configuration}\label{sec:op-config}

The operational configuration uses a YAML file mapped at \texttt{/opt/cxone/config.yaml}
by default.  It is possible to map the \texttt{config.yaml} file to another location in the
container and adjust the path via the \texttt{CONFIG\_YAML\_PATH} environment variable.

\subsection{YAML Configuration Examples}

\subsubsection{Basic YAML Configurations}

The following example shows a minimal \cxoneflow configuration that defines the following:

\begin{enumerate}
    \item Files containing secrets are located at \texttt{/run/secrets}.
    \item One BitBucket Data Center SCM connection configuration to handle all webhook payloads
    POSTed to the \texttt{/bbdc} endpoint.
    \item One catch-all route for clone-urls using the regular expression \texttt{.*}
    \item The SCM's base URL located at \texttt{https://scm.corp.com}
    \item The shared secret used to validate webhook payloads located in the file \texttt{/run/secrets/scm-shared-secret}
    \item The API and clone authorization using a PAT in a file located at \texttt{/run/secrets/scm-token-secret}
    \item The CheckmarxOne tenant name of \texttt{mytenant}
    \item The CheckmarxOne API credentials using an \texttt{oauth} client with:
    \begin{enumerate}
        \item The client identifier located in the file \texttt{/run/secrets/my-oauth-id}
        \item The client secret located in the file \texttt{/run/secrets/my-oauth-secret}
    \end{enumerate}
    \item Using the CheckmarxOne multi-tenant US region IAM endpoint.
    \item Using the CheckmarxOne multi-tenant US region API endpoint.
\end{enumerate}

\begin{code}{Minimal YAML Configuration Example \#1}{Using BitBucket Data Center}{}
secret-root-path: /run/secrets
server-base-url: https://cxoneflow.mydomain.com:8443/

bbdc:
    - service-name: BitBucket DC
      repo-match: .*
      connection:
        base-url: https://scm.corp.com
        shared-secret: scm-shared-secret
        api-auth:
            token: scm-token-secret
      cxone:
        tenant: mytenant
        oauth:
            client-id: my-oauth-id
            client-secret: my-oauth-secret
        iam-endpoint: US
        api-endpoint: US
\end{code}

\pagebreak
\noindent\\An alternate minimal example using different authorization options:

\begin{code}{Minimal YAML Configuration Example \#2}{Using BitBucket Data Center}{}
secret-root-path: /run/secrets
server-base-url: https://cxoneflow.mydomain.com:8443/

bbdc:
    - service-name: BitBucket DC
      repo-match: .*
      connection:
      base-url: https://scm.corp.com
      shared-secret: scm-shared-secret
      api-auth:
          username: scm-username-secret
          password: scm-password-secret
      clone-auth:
          ssh: scm-ssh-key-secret
      cxone:
        tenant: mytenant
        api-key: my-cxone-api-key
        iam-endpoint: US
        api-endpoint: US
\end{code}
    
\pagebreak
\noindent\\An alternate minimal example using for an Azure DevOps Enterprise
SCM:

\begin{code}{Minimal YAML Configuration Example \#2}{Using Azure DevOps Enterprise}{}
secret-root-path: /run/secrets
server-base-url: https://cxoneflow.mydomain.com:8443/

adoe:
    - service-name: MyADO
      repo-match: .*
      connection:
      base-url: https://scm.corp.com/DefaultCollection
      shared-secret: scm-shared-secret
      api-auth:
          username: scm-username-secret
          password: scm-password-secret
      clone-auth:
          ssh: scm-ssh-key-secret
      cxone:
        tenant: mytenant
        api-key: my-cxone-api-key
        iam-endpoint: US
        api-endpoint: US
\end{code}


\begin{code}{Minimal YAML Configuration Example}{Enabling Pull Request Feedback}{}
    secret-root-path: /run/secrets
    server-base-url: https://cxoneflow.mydomain.com:8443/

    bbdc:
        - service-name: BitBucket DC
          repo-match: .*
          feedback:
            pull-request:
              enabled: "True"
          connection:
            base-url: https://scm.corp.com
            shared-secret: scm-shared-secret
            api-auth:
                token: scm-token-secret
          cxone:
            tenant: mytenant
            oauth:
                client-id: my-oauth-id
                client-secret: my-oauth-secret
            iam-endpoint: US
            api-endpoint: US
    \end{code}
    

\newpage
This example shows a \cxoneflow configuration using all default options defined
as a handler when using a single SCM.  This is the same as one of the simple examples if
an example had explicitly defined omitted default connection elements.

The \texttt{scan-config} element has been added to this configuration to
demonstrate some of the controls that can be implemented over scan options.  In this
example, static Project and Scan tags are defined.  Also defined is the selection
of engines for the scan with some options defined as supported by the engine.  The
option keys are defined by the
\href{https://checkmarx.stoplight.io/docs/checkmarx-one-api-reference-guide/branches/main/f601dd9456e80-run-a-scan}{Scan REST API}
and described by
\href{https://checkmarx.com/resource/documents/en/34965-68598-global-settings.html#UUID-8e38f06b-45d4-ea7f-5ff5-50deb22e43aa_UUID-1a4211ec-dbf9-a180-cb20-59e1246ec3fb}{Scanners Settings}.


\begin{code}{Full YAML Configuration Example}{[CxOne: oauth]}{[SCM: token auth]}
secret-root-path: /run/secrets
server-base-url: https://cxoneflow.mydomain.com:8443/

bbdc:
    - service-name: BitBucket DC
      repo-match: .*
      scan-config:
          default-scan-engines:
              sca:
                  exploitablePath: "True"
              sast:
                  presetName: ASA Premium
                  incremental: "False"
                  fastScanMode: "True"
                  filter: "!**/node_modules,!**/test*"
                  languageMode: multi
              kics:
              apisec:
          default-scan-tags:
              scan-service: BitBucket DC
          default-project-tags:
              onboarded-by: CxOneFlow
      connection:
          base-url: https://scm.corp.com
          shared-secret: scm-shared-secret
          timeout-seconds: 60
          retries: 3
          proxies:
            http: http://proxy.corp.com:8080
            https: http://proxy.corp.com:8080
          api-auth:
              token: scm-token-secret
      cxone:
          tenant: mytenant
          oauth:
              client-id: my-oauth-id
              client-secret: my-oauth-secret
          iam-endpoint: US
          api-endpoint: US
          timeout-seconds: 60
          retries: 3
          proxies:
            http: http://proxy.corp.com:8080
            https: http://proxy.corp.com:8080
\end{code}

\pagebreak
\noindent\\The next example shows a configuration where \cxoneflow has endpoint handlers for both
BitBucket Data Center and Azure DevOps Enterprise.  Each SCM is configured to handle 3 distinct
projects to demonstrate Basic, Token, and SSH authentication methods.  All the SCM endpoints
orchestrate scans in a single Checkmarx One tenant.

\begin{code}{Multi-SCM YAML Configuration Example}{}{}
secret-root-path: /run/secrets
server-base-url: https://cxoneflow.mydomain.com:8443/
adoe-connection: &adoe-con
    base-url: http://adoe.scm.org/DefaultCollection
    shared-secret: scm-shared-secret
bbdc-connection: &bbdc-con
    base-url: http://bbdc.scm.org
    shared-secret: scm-shared-secret
adoe:
    - service-name: ADO-AsSSH
        repo-match: .*MySSHProject
        connection:
        <<: *adoe-con
        api-auth: 
            token: adoe-token-secret
        clone-auth: &clone-ssh
            ssh: ssh-priv-key
        cxone: &cxone
        tenant: my_tenant
        oauth:
            client-id: prod_client_id
            client-secret: prod_client_secret
        iam-endpoint: US
        api-endpoint: US
    - service-name: ADO-AsToken
        repo-match: .*MyTokenProject
        connection:
        <<: *adoe-con
        api-auth:
            token: adoe-token-secret
        cxone: *cxone
    - service-name: ADO-AsBasicAuth
        repo-match: .*MyBasicAuthProject
        connection:
        <<: *adoe-con
        api-auth:
            token: adoe-token-secret
        clone-auth: &basic-creds
            username: username-secret
            password: password-secret
        cxone: *cxone
bbdc:
    - service-name: BBDC-AsSSH
        repo-match: .*SP
        connection:
        <<: *bbdc-con
        api-auth: 
            token: bbdc-ssh-proj-pat
        clone-auth: *clone-ssh
        cxone: *cxone
    - service-name: BBDC-AsToken
        repo-match: .*TOK
        connection:
        <<: *bbdc-con
        api-auth:
            token: bbdc-token-proj-pat
        cxone: *cxone
    - service-name: BBDC-AsBasicAuth
        repo-match: .*BAS
        connection:
        <<: *bbdc-con
        api-auth: *basic-creds
        cxone: *cxone
\end{code}



\subsubsection{Complex YAML Configurations using YAML Anchors}

For complex configurations, it is possible to use 
\extlink{https://docs.docker.com/compose/compose-file/10-fragments/}{YAML Anchors}
to avoid repeating some section definitions.  When using YAML anchors, it may be useful
to use a \extlink{https://onlineyamltools.com/convert-yaml-to-json}{YAML-to-JSON} conversion tool that shows the JSON generated from the YAML
definition

This example demonstrates defining common connection parameters that can be applied
to all connection definitions:

\begin{code}{Compacted Full YAML Configuration Example}{[CxOne: oauth]}{[SCM: token auth]}
secret-root-path: /run/secrets
server-base-url: https://cxoneflow.mydomain.com:8443/
my-connection-params: &common-connection-params
    timeout-seconds: 60
    retries: 3
    proxies:
        http: http://proxy.corp.com:8080
        https: http://proxy.corp.com:8080


bbdc:
    - service-name: BitBucket DC
      repo-match: .*
      feedback:
        pull-request:
            enabled: "True"
      scan-config:
          default-scan-engines:
              sca:
                  exploitablePath: "True"
              sast:
                  presetName: ASA Premium
                  incremental: "False"
                  fastScanMode: "True"
                  filter: "!**/node_modules,!**/test*"
                  languageMode: multi
              kics:
              apisec:
          default-scan-tags:
              scan-service: BitBucket DC
          default-project-tags:
              onboarded-by: CxOneFlow
      connection:
          base-url: https://scm.corp.com
          shared-secret: scm-shared-secret
          api-auth:
              token: scm-token-secret
          <<: *common-connection-params
      cxone:
          tenant: mytenant
          oauth:
              client-id: my-oauth-id
              client-secret: my-oauth-secret
          iam-endpoint: US
          api-endpoint: US
          <<: *common-connection-params
\end{code}


\noindent\\It is common to see a scenario where there are multiple organizations
using the same SCM instance.  A single \cxoneflow instance can be configured to accept
webhook events from all repos in each organization by using the \texttt{repo-match}
regular expression.  When a webhook payload is received, the \texttt{repo-match}
regular expression is applied to the clone URI until a match is found.

\noindent\\The example YAML below is used to demonstrate how \cxoneflow could be configured
for mulitple organizations in a single SCM. In the example, YAML anchors are utilized to 
re-use the common settings for each SCM organization.  Each organization, in this case, 
exists in the same SCM server and shares the same Checkmarx One instance.

\begin{code}{SCM Multi-Org YAML Configuration Example}{[CxOne: oauth]}{[SCM: token auth]}
secret-root-path: /run/secrets
server-base-url: https://cxoneflow.mydomain.com:8443/
my-connection-params: &common-connection-params
    timeout-seconds: 60
    retries: 3
    proxies:
    http: http://proxy.corp.com:8080
    https: http://proxy.corp.com:8080

bbdc:
    - service-name: BBDC-West
      repo-match: .*west
      scan-config: 
          default-scan-engines: &common-engine-config
              sca:
                  exploitablePath: "True"
              sast:
                  presetName: ASA Premium
                  incremental: "False"
                  fastScanMode: "True"
                  filter: "!**/node_modules,!**/test*"
                  languageMode: multi
              kics:
              apisec:
          default-scan-tags:
              scan-service: BBDC-West
          default-project-tags:
              onboarded-by: CxOneFlow
              region: West
      connection:
          base-url: https://scm.corp.com
          shared-secret: scm-west-org-shared-secret
          api-auth:
              token: scm-token-secret
          <<: *common-connection-params
      cxone: &cxone-config
          tenant: mytenant
          oauth:
              client-id: my-oauth-id
              client-secret: my-oauth-secret
          iam-endpoint: US
          api-endpoint: US
          <<: *common-connection-params
    - service-name: BBDC-East
      repo-match: .*east
      scan-config: 
          default-scan-engines: *common-engine-config
          default-scan-tags:
              scan-service: BBDC-East
          default-project-tags:
              onboarded-by: CxOneFlow
              region: East
      connection:
          base-url: https://scm.corp.com
          shared-secret: scm-east-org-shared-secret
          api-auth:
              token: scm-token-secret
          <<: *common-connection-params
      cxone: *cxone-config
\end{code}



\subsection{YAML Configuration Elements}\label{sec:yaml-config}

The organization of the YAML configuration is depicted in the tree below.  The description of each element
can be referenced by clicking the element.  Required elements are indicated in the tree; in general, if an
element that is not marked "required" is omitted, the feature that performs that operation is not invoked
for the configured service definition.

The \texttt{<root>} element indicates that elements directly under the root start at the farthest
left index of the line (this means a line position with an index of 0).  YAML elements that appear
under a parent element are intended to start at first tab stop past the parent element's tab stop.
Anchor elements may be defined at the root but must not clash with the names of any of the root elements.

Parts of the YAML tree have been split into individual trees to allow related elements to appear together.

\paragraph{YAML Root Elements}\label{sec:yaml-root}

\noindent\\

\dirtree{%
    .1 <root>.
    .2 \intlink{sec:yaml-secret-root-path}{secret-root-path} \DTcomment{[Required]}.
    .2 \intlink{sec:yaml-server-base-url}{server-base-url} \DTcomment{[Required]}.
    .2 \intlink{sec:yaml-scm-monikers}{<scm moniker>} \DTcomment{[At least 1 required: \textbf{bbdc}, \textbf{adoe}, \textbf{gh}]}.
    .3 \intlink{sec:moniker-elements}{...see "YAML SCM Moniker Elements"}.
}

\pagebreak
\paragraph{YAML SCM Moniker Elements}\label{sec:moniker-elements}
\noindent\\The \texttt{<scm moniker>} element is a YAML list of dictionaries.  For a YAML list,
each entry is indented to the next tab after the parent and prefixed with a "\texttt{-}" (dash).
The elements in each list entry under \texttt{<scm moniker>} define key/value dictionary entries 
as the list entry.  Each list entry is referred to as a "service definition"
elsewhere in this document.\\\\

\dirtree{%
    .1 <root>.
    .2 \intlink{sec:yaml-scm-monikers}{<scm moniker>} \DTcomment{[Limited to: \textbf{bbdc}, \textbf{adoe}, \textbf{gh}]}.
    .3 \intlink{sec:yaml-moniker-connection}{connection} \DTcomment{[Required]}.
    .4 \intlink{sec:connection-elements}{...see "YAML \texttt{connection} Elements"}.
    .3 \intlink{sec:yaml-moniker-cxone}{cxone} \DTcomment{[Required]}.
    .4 \intlink{sec:cxone-elements}{...see "YAML \texttt{cxone} Elements"}.
    .3 \intlink{sec:yaml-moniker-feedback}{feedback} \DTcomment{[Optional]}.
    .4 \intlink{sec:feedback-elements}{...see "YAML \texttt{feedback} Elements"}.
    .3 \intlink{sec:yaml-moniker-repo-match}{repo-match} \DTcomment{[Required]}.
    .3 \intlink{sec:yaml-moniker-resolver}{resolver} \DTcomment{[Optional]}.
    .4 \intlink{sec:resolver-elements}{...see "YAML \texttt{resolver} Elements"}.
    .3 \intlink{sec:yaml-moniker-scan-config}{scan-config} \DTcomment{[Optional]}.
    .4 \intlink{sec:scan-config-elements}{...see "YAML \texttt{scan-config} Elements"}.
    .3 \intlink{sec:yaml-moniker-service-name}{service-name} \DTcomment{[Required]}.
}

\pagebreak
\paragraph{YAML \texttt{connection} Elements}\label{sec:connection-elements}
\noindent\\

\dirtree{%
    .1 <root>.
    .2 \intlink{sec:yaml-scm-monikers}{<scm moniker>} \DTcomment{[Required: \textbf{bbdc}, \textbf{adoe}, \textbf{gh}]}.
    .3 \intlink{sec:yaml-moniker-connection}{connection} \DTcomment{[Required]}.
    .4 \intlink{sec:yaml-connection-base-url}{base-url} \DTcomment{[Required]}.
    .4 \intlink{sec:yaml-connection-base-display-url}{base-display-url} \DTcomment{[Required for some SCMs]}.
    .4 \intlink{sec:yaml-connection-api-url-suffix}{api-url-suffix} \DTcomment{[Required for some SCMs]}.
    .4 \intlink{sec:yaml-connection-shared-secret}{shared-secret} \DTcomment{[Required]}.
    .4 \intlink{sec:yaml-generic-proxies}{proxies} \DTcomment{[Optional]}.
    .4 \intlink{sec:yaml-generic-retries}{retries} \DTcomment{[Optional] Default: 3}.
    .4 \intlink{sec:yaml-generic-ssl-verify}{ssl-verify} \DTcomment{[Optional] Default: True}.
    .4 \intlink{sec:yaml-generic-timeout-seconds}{timeout-seconds}\DTcomment{[Optional] Default: 60s}.
    .4 \intlink{sec:yaml-connection-api-auth}{api-auth} \DTcomment{[Required]}.
    .5 \intlink{sec:yaml-api-auth-app-private-key}{app-private-key} \DTcomment{[See element documentation]}.
    .5 \intlink{sec:yaml-api-auth-password}{password} \DTcomment{[See element documentation]}.
    .5 \intlink{sec:yaml-api-auth-token}{token} \DTcomment{[See element documentation]}.
    .5 \intlink{sec:yaml-api-auth-username}{username} \DTcomment{[See element documentation]}.
    .4 \intlink{sec:yaml-connection-clone-auth}{clone-auth} \DTcomment{[Optional] Default: \texttt{api-auth}}.
    .5 \intlink{sec:yaml-api-auth-password}{password} \DTcomment{[See element documentation]}.
    .5 \intlink{sec:yaml-clone-auth-ssh}{ssh} \DTcomment{[See element documentation]}.
    .5 \intlink{sec:yaml-clone-auth-ssh-port}{ssh-port} \DTcomment{[See element documentation]}.
    .5 \intlink{sec:yaml-api-auth-token}{token} \DTcomment{[See element documentation]}.
    .5 \intlink{sec:yaml-api-auth-username}{username} \DTcomment{[See element documentation]}.
}

\pagebreak
\paragraph{YAML \texttt{cxone} Elements}\label{sec:cxone-elements}
\noindent\\

\dirtree{%
    .1 <root>.
    .2 \intlink{sec:yaml-scm-monikers}{<scm moniker>} \DTcomment{[Required: \textbf{bbdc}, \textbf{adoe}, \textbf{gh}]}.
    .3 \intlink{sec:yaml-moniker-cxone}{cxone} \DTcomment{[Required]}.
    .4 \intlink{sec:yaml-cxone-api-endpoint}{api-endpoint} \DTcomment{[Required]}.
    .4 \intlink{sec:yaml-cxone-api-key}{api-key} \DTcomment{[Required without oauth]}.
    .4 \intlink{sec:yaml-cxone-iam-endpoint}{iam-endpoint} \DTcomment{[Required]}.
    .4 \intlink{sec:yaml-cxone-oauth}{oauth} \DTcomment{[Required without api-key]}.
    .4 \intlink{sec:yaml-generic-proxies}{proxies} \DTcomment{[Optional]}.
    .4 \intlink{sec:yaml-generic-retries}{retries} \DTcomment{[Optional] Default: 3}.
    .4 \intlink{sec:yaml-generic-ssl-verify}{ssl-verify} \DTcomment{[Optional] Default: True}.
    .4 \intlink{sec:yaml-cxone-tenant}{tenant} \DTcomment{[Required]}.
    .4 \intlink{sec:yaml-generic-timeout-seconds}{timeout-seconds}\DTcomment{[Optional] Default: 60s}.
}


\pagebreak
\paragraph{YAML \texttt{feedback} Elements}\label{sec:feedback-elements}
\noindent\\


\dirtree{%
    .1 <root>.
    .2 \intlink{sec:yaml-scm-monikers}{<scm moniker>} \DTcomment{[Required: \textbf{bbdc}, \textbf{adoe}, \textbf{gh}]}.
    .3 \intlink{sec:yaml-moniker-feedback}{feedback} \DTcomment{[Optional]}.
    .4 \intlink{sec:yaml-generic-amqp}{amqp} \DTcomment{[Optional] Default: container instance}.
    .5 \intlink{sec:yaml-generic-amqp-amqp-password}{amqp-password} \DTcomment{[Optional]}.
    .5 \intlink{sec:yaml-generic-amqp-amqp-url}{amqp-url} \DTcomment{[Required]}.
    .5 \intlink{sec:yaml-generic-amqp-amqp-user}{amqp-user} \DTcomment{[Optional]}.
    .5 \intlink{sec:yaml-generic-ssl-verify}{ssl-verify} \DTcomment{[Optional] Default: True}.
    .4 \intlink{sec:yaml-feedback-pull-request}{pull-request} \DTcomment{[Optional]}.
    .5 \intlink{sec:yaml-pull-request-enabled}{enabled} \DTcomment{[Optional] Default: False}.
    .4 \intlink{sec:yaml-feedback-scan-monitor}{scan-monitor} \DTcomment{[Optional]}.
    .5 \intlink{sec:yaml-scan-monitor-poll-backoff-multiplier}{poll-backoff-multiplier} \DTcomment{[Optional] Default: 2}.
    .5 \intlink{sec:yaml-scan-monitor-poll-interval-seconds}{poll-interval-seconds} \DTcomment{[Optional] Default: 90s}.
    .5 \intlink{sec:yaml-scan-monitor-poll-max-interval-seconds}{poll-max-interval-seconds} \DTcomment{[Optional] Default: 600s}.
    .5 \intlink{sec:yaml-scan-monitor-scan-timeout-hours}{scan-timeout-hours} \DTcomment{[Optional] Default: 48h}.
    .4 \intlink{sec:yaml-feedback-exclusions}{exclusions} \DTcomment{[Optional]}.
    .5 \intlink{sec:yaml-exclusions-severity}{severity} \DTcomment{[Optional]}.
    .5 \intlink{sec:yaml-exclusions-state}{state} \DTcomment{[Optional]}.
}


\pagebreak
\paragraph{YAML \texttt{resolver} Elements}\label{sec:resolver-elements}
\noindent\\

\dirtree{%
    .1 <root>.
    .2 \intlink{sec:yaml-scm-monikers}{<scm moniker>} \DTcomment{[Required: \textbf{bbdc}, \textbf{adoe}, \textbf{gh}]}.
    .3 \intlink{sec:yaml-moniker-resolver}{resolver} \DTcomment{[Optional]}.
    .4 \intlink{sec:yaml-generic-amqp}{amqp} \DTcomment{[Optional] Default: container instance}.
    .5 \intlink{sec:yaml-generic-amqp-amqp-password}{amqp-password} \DTcomment{[Optional]}.
    .5 \intlink{sec:yaml-generic-amqp-amqp-url}{amqp-url} \DTcomment{[Required]}.
    .5 \intlink{sec:yaml-generic-amqp-amqp-user}{amqp-user} \DTcomment{[Optional]}.
    .5 \intlink{sec:yaml-generic-ssl-verify}{ssl-verify} \DTcomment{[Optional] Default: True}.
    .3 \intlink{sec:yaml-resolver-allowed-agent-tags}{allowed-agent-tags} \DTcomment{[Required]}.
    .3 \intlink{sec:yaml-resolver-capture-resolver-logs}{capture-resolver-logs} \DTcomment{[Optional] Default: False}.
    .3 \intlink{sec:yaml-resolver-default-agent-tag}{default-agent-tag} \DTcomment{[Optional]}.
    .3 \intlink{sec:yaml-resolver-private-key}{private-key} \DTcomment{[Required]}.
    .3 \intlink{sec:yaml-resolver-resolver-tag-key}{resolver-tag-key} \DTcomment{[Optional] Default: resolver}.
    .3 \intlink{sec:resolver-scan-retries}{scan-retries}\DTcomment{[Optional] Default: 3}.
    .3 \intlink{sec:resolver-scan-timeout-seconds}]{scan-timeout-seconds}\DTcomment{[Optional] Default: 10800}.
}


\pagebreak
\paragraph{YAML \texttt{scan-config} Elements}\label{sec:scan-config-elements}
\noindent\\

\dirtree{%
    .1 <root>.
    .2 \intlink{sec:yaml-scm-monikers}{<scm moniker>} \DTcomment{[Required: \textbf{bbdc}, \textbf{adoe}, \textbf{gh}]}.
    .3 \intlink{sec:yaml-moniker-scan-config}{scan-config} \DTcomment{[Optional]}.
    .4 \intlink{sec:yaml-scan-config-default-scan-engines}{default-scan-engines} \DTcomment{[Optional]}.
    .4 \intlink{sec:yaml-scan-config-default-project-tags}{default-project-tags} \DTcomment{[Optional]}.
    .4 \intlink{sec:yaml-scan-config-default-scan-tags}{default-scan-tags} \DTcomment{[Optional]}.
}



\subsubsection{YAML Element: secret-root-path}\label{sec:yaml-secret-root-path}

A string that is the path to a directory that contains one or more files containing secret values.  The names to these files are 
referenced elsewhere in the YAML configuration file when used in a field that is a reference to a secret.  For most deployment
purposes, this is a path to a location in the \cxoneflow running container.

\subsubsection{YAML Element: server-base-url}\label{sec:yaml-server-base-url}
A string that is the base URL for the \cxoneflow endpoint.  This is used when creating feedback content that loads image elements.

\subsubsection{YAML Element: <scm moniker>}\label{sec:yaml-scm-monikers}

This is a moniker indicating the a list of service definitions for handling events from an SCM matching the name of the SCM
moniker.  Each service definition is a YAML dictionary of elements. The contents for each service definition dictionary 
have the same meaning unless otherwise specified.  At lease one SCM moniker with one configured service definition is required. 
The following SCM configuration monikers are currently supported:

\begin{itemize}
    \item \textbf{\texttt{bbdc}} for BitBucket Data Center webhook payloads targeting the \texttt{/bbdc}
    webhook payload receiver endpoint.
    \item \textbf{\texttt{adoe}} for Azure DevOps Enterprise or Cloud webhook payloads targeting the \texttt{/adoe}
    webhook payload receiver endpoint.
    \item \textbf{\texttt{gh}} for GitHub Enterprise or Cloud webhook payloads targeting the \texttt{/gh}
    webhook payload receiver endpoint.
\end{itemize}


\subsubsection{YAML Element: scan-config}\label{sec:yaml-moniker-scan-config}
A block element where the child elements define the default scan configuration for this service endpoint.

\subsubsection{YAML Element: service-name}\label{sec:yaml-moniker-service-name}
A moniker for the service definition. The moniker is used for logging and workflow purposes.

\subsubsection{YAML Element: repo-match}\label{sec:yaml-moniker-repo-match}
A regex applied to the source repository.  If the webhook payload has
a clone URL that matches the regex, this service definition is used to orchestrate scanning
for the received event.

\subsubsection{YAML Element: connection}\label{sec:yaml-moniker-connection}
A block element where the child elements define the SCM connection parameters for this service definition.

\subsubsection{YAML Element: cxone}\label{sec:yaml-moniker-cxone}
A block element where the child elements define the connection configuration for the \cxone API. 

\subsubsection{YAML Element: feedback}\label{sec:yaml-moniker-feedback}
A block element where the child elements define the configuration for feedback workflows. 

\subsubsection{YAML Element: resolver}\label{sec:yaml-moniker-resolver}
A block element where the child elements define the configuration for support of resolver agents. 


\subsubsection{YAML Element: default-scan-engines}\label{sec:yaml-scan-config-default-scan-engines}

A dictionary YAML element that follows the format \texttt{<engine-name>:<engine config option dictionary>}
corresponding to the configuration element of the
\extlink{https://checkmarx.stoplight.io/docs/checkmarx-one-api-reference-guide/branches/main/f601dd9456e80-run-a-scan}{\cxonetext scan API}.
Any static engine configuration supported by an engine can be defined here.  If a configuration is defined that is not supported
by the engine, scans may fail to start.

\subsubsection{YAML Element: default-scan-tags}\label{sec:yaml-scan-config-default-scan-tags}
A dictionary YAML element containing static key:value pairs that are assigned to each scan.

\subsubsection{YAML Element: default-project-tags}\label{sec:yaml-scan-config-default-project-tags}
A dictionary YAML element containing static key:value pairs that are assigned\\to each project upon project creation.


\subsubsection{YAML Element: tenant}\label{sec:yaml-cxone-tenant}
The name of the \cxone tenant.

\subsubsection{YAML Element: iam-endpoint}\label{sec:yaml-cxone-iam-endpoint}
This can be a fully qualified domain name of a server or a multi-tenant IAM endpoint moniker as described in Appendix \ref{sec:endpoint-monikers}.

\subsubsection{YAML Element: api-endpoint}\label{sec:yaml-cxone-api-endpoint}
This can be a fully qualified domain name of a server or a multi-tenant API endpoint moniker as described in Appendix \ref{sec:endpoint-monikers}.

\subsubsection{YAML Element: api-key}\label{sec:yaml-cxone-api-key}
If not defined, the \texttt{oauth} element must be defined. The value specifies a file name found under the path defined by \texttt{secret-root-path}.

\subsubsection{YAML Element: oauth}\label{sec:yaml-cxone-oauth}
If not defined, the \texttt{api-key} element must be defined. This contains two required elements
\texttt{client-id} and \texttt{client-secret} where each value corresponds to a file name found under the path defined by \texttt{secret-root-path}. 


\subsubsection{YAML Element: pull-request}\label{sec:yaml-feedback-pull-request}
The configuration parameters for pull request feedback workflows.

\subsubsection{YAML Element: scan-monitor}\label{sec:yaml-feedback-scan-monitor}
The parameters used when monitoring scan progress during workflow orchestration. 

Scan progress is monitored by requesting a scan state from the \cxone API at
a time interval.  The initial time interval is set to the value configured for
\texttt{poll-interval-seconds}.  If the scan is not found to have finished executing
at any given poll execution, the previous poll interval time is multiplied by
the scalar given in the \texttt{poll-backoff-multiplier} value up to a maximum
poll interval time configured by \texttt{poll-max-interval-seconds}.

If a scan does not finish executing by the time set in \texttt{scan-timeout-hours}, the
workflow is aborted.  The value of 0 configured for \texttt{scan-timeout-hours} indicates
the workflow will wait forever for the scan to finish executing.

\subsubsection{YAML Element: exclusions}\label{sec:yaml-feedback-exclusions}
Settings for excluding results from results from feedback output.  Each of the elements is a list that 
can be configured with multiple exclusion elements.

\subsubsection{YAML Element: enabled}\label{sec:yaml-pull-request-enabled}
Defaults to \texttt{False}.  If set to \texttt{True}, the feedback workflow for Pull Requests is executed upon completion of a scan generated by
a Pull Request. See Section \ref{sec:pull-request-workflow} for details about the Pull Request feedback workflow.




\subsubsection{YAML Element: poll-interval-seconds}\label{sec:yaml-scan-monitor-poll-interval-seconds}
The number of seconds to use in calculating scan status polling time intervals.


\subsubsection{YAML Element: poll-max-interval-seconds}\label{sec:yaml-scan-monitor-poll-max-interval-seconds}
The maximum polling interval seconds.


\subsubsection{YAML Element: poll-backoff-multiplier}\label{sec:yaml-scan-monitor-poll-backoff-multiplier}
A scalar used to increase the scan polling interval after each poll execution.

\subsubsection{YAML Element: scan-timeout-hours}\label{sec:yaml-scan-monitor-scan-timeout-hours}
The number of hours before a feedback workflow aborts waiting for a scan to finish executing.  Set to 0 to wait forever.

\subsubsection{YAML Element: state }\label{sec:yaml-exclusions-state}
A YAML list element that supports the following values:

\begin{itemize}
  \item Not Exploitable
  \item To Verify
  \item Proposed Not Exploitable
  \item Confirmed
  \item Urgent
\end{itemize}



\subsubsection{YAML Element: severity}\label{sec:yaml-exclusions-severity}
A YAML list element that supports the following values:

\begin{itemize}
  \item Critical
  \item High
  \item Medium
  \item Low
  \item Info
\end{itemize}



\subsubsection{YAML Element: base-url}\label{sec:yaml-connection-base-url}
The base url of the SCM server's API endpoint. This should be the root URL for the API that can be used when composing all
API calls related to the source of the received webhook event.

\subsubsection{YAML Element: base-display-url}\label{sec:yaml-connection-base-display-url}
An optional URL for use when composing links as part of an information display such as pull-request
feedback. Most SCMs will not require this setting.

\subsubsection{YAML Element: api-url-suffix}\label{sec:yaml-connection-api-url-suffix}
An optional API URL suffix used when composing API request URLs. Most SCMs will not require this setting.

\subsubsection{YAML Element: shared-secret}\label{sec:yaml-connection-shared-secret}
The shared secret configured in the SCM used to sign webhook payloads. The shared secret must meet the
following minimum criteria: 

\begin{itemize}
  \item 20 characters long
  \item contains at least 3 numbers
  \item contains at least 3 upper-case letters
  \item contains at least 2 special characters
\end{itemize}

\subsubsection{YAML Element: api-auth}\label{sec:yaml-connection-api-auth}
A YAML dictionary of SCM authorization options for using the API.

The authorization methods for \texttt{api-auth} 
are used to communicate with the SCM's API and can often be used for cloning repositories.  The
main difference between \texttt{api-auth} and \texttt{clone-auth} is that API access generally
does not support SSH authorization. If there is a need to clone using SSH, configure the SSH
authorization under the \texttt{clone-auth} element.  

The elements of \texttt{api-auth} are required depending on the type of authorization that
needs to be performed.

\subsubsection{YAML Element: clone-auth}\label{sec:yaml-connection-clone-auth}
Defines authorization options for performing clones when it differs from authorization for API requests.

The \texttt{clone-auth} element is optional;  if not provided, the connection information defined
in \texttt{api-auth} will be used.
\subsubsection{YAML Element: token}\label{sec:yaml-api-auth-token}
The value specifies a file name found under the path defined by \texttt{secret-root-path} 
containing a Personal Access Token (PAT).  This is required for token authorization.  This can
be combined with the \texttt{username} element.


\subsubsection{YAML Element: username}\label{sec:yaml-api-auth-username}

\textbf{For Token Authorization}
The value specifies a file name found under the path defined by \texttt{secret-root-path} containing a username associated with the PAT.
This is optional and only used during cloning; if not supplied, the default username of \texttt{git} is used. Can be combined with 
the \texttt{token} element.


\textbf{For Basic Authorization}\footnote{Many SCMs no longer support basic authorization.}
The value specifies a file name found under the path defined by \texttt{secret-root-path} containing the username associated with the account used
for authorization.  This element can be supplied with the \texttt{password} element.

\subsubsection{YAML Element: password}\label{sec:yaml-api-auth-password}
The value specifies a file name found under the path defined by \texttt{secret-root-path} 
containing a password associated with the username found in the \texttt{username} element.  
This is required for basic authorization if basic authorization is supported by the SCM instance.
This can be combined with the \texttt{username} element.


\subsubsection{YAML Element: app-private-key}\label{sec:yaml-api-auth-app-private-key}
The value specifies a file name found under the path defined by \texttt{secret-root-path} containing a private key used
when obtaining application authorization. 

Application Authorization is available for use with select SCM types. Refer to Part \ref{part:scms} for details about SCMs
that support this type of authorization. When using Application Authorization, there is typically not a need to provide a separate 
method of authorization for cloning defined in the \texttt{clone-auth} element. 

\subsubsection{YAML Element: ssh}\label{sec:yaml-clone-auth-ssh}
The value specifies a file name found under the path defined by \texttt{secret-root-path} containing an unencrypted private SSH key.

\subsubsection{YAML Element: ssh-port}\label{sec:yaml-clone-auth-ssh-port}
This optional value specifies the port used for SSH cloning if the SCM is not using port 22
and does not automatically include it in the clone URL.


\chapter{Resolver Workflows}\label{sec:resolver-workflows}

\section{Overview}
For any scan invoked by \cxoneflow\space when \intlink{sec:resolver-elements}{configured to use resolver}, 
a "deferred" scan is enqueued with the message queue to run a resolver scan.  The deferred scan is handed over
to a distributed resolver agent that clones the code and executes \scaresolver.  The \scaresolver results are
then sent back to the \cxoneflow endpoint server to submit for scan along with the code.  This section details
exchange, queue, and workflow logic of this process.


\section{Deferred Scan Workflow}

The \cxoneflow endpoint logs will indicate scans are deferred when the algorithm for event handling
determines a resolver scan is needed.  Figure \ref{fig:deferred-scan-flowchart} shows the deferred-scan
workflow algorithm that is followed to determine when to request a deferred scan.  

\begin{figure}[ht]
  \includegraphics[width=\textwidth]{graphics/cxoneflow-diagrams-Deferred Scan Algorithm.png}
  \caption{Deferred Scan Algorithm}
  \label{fig:deferred-scan-flowchart}
\end{figure}


\section{Post Deferred Scan Workflow}

When the \scaresolver scan is complete, the distributed resolver agent will send the results
to the \cxoneflow endpoint for submission in the \cxone scan.  Scans with \scaresolver can
complete successfully or end with failure.

If the \scaresolver scan is successful, the results of the 
\extlink{https://docs.checkmarx.com/en/34965-19199-running-scans-using-checkmarx-sca-resolver.html\#UUID-af718204-6dfc-2b27-439e-419b9157d364_id_RunningScansUsingCheckmarxSCAResolver-CheckmarxSCAResolverModes}{offline}
scan are submitted as part of the \cxone scan.  If the \scaresolver scan fails (for any reason), the \cxone scan is submitted for server-side dependency resolution.
Figure \ref{fig:post-deferred-scan-flowchart} shows the post-deferred-scan workflow algorithm.

\begin{figure}[ht]
  \includegraphics[width=\textwidth]{graphics/cxoneflow-diagrams-Post Deferred Scan Algorithm.png}
  \caption{Post Deferred Scan Algorithm}
  \label{fig:post-deferred-scan-flowchart}
\end{figure}




\subsubsection{YAML Element: amqp}\label{sec:yaml-generic-amqp}
The connection parameters for an AMQP endpoint used for workflow orchestration and resolver agent coordination. 
If not set, the internal RabbitMQ instance in the \cxoneflow container is used by default.

\subsubsection{YAML Element: amqp-url}\label{sec:yaml-generic-amqp-amqp-url}
This can be one of the following values:

\begin{itemize}
  \item The AMQP/AMQPS URL for the AMQP endpoint.
  \item The name of a file container a secret located at the path defined by \texttt{secret-root-path}.
\end{itemize}


\subsubsection{YAML Element: amqp-user}\label{sec:yaml-generic-amqp-amqp-user}
If the user name is not included in the AMQP URL, the provided value corresponds to a file name found under
the path defined by \texttt{secret-root-path}.

\subsubsection{YAML Element: amqp-password}\label{sec:yaml-generic-amqp-amqp-password}
If the password is not included in the AMQP URL, the provided value corresponds to a file name found
under the path defined by \texttt{secret-root-path}.


\subsubsection{YAML Element: ssl-verify}\label{sec:yaml-generic-ssl-verify}
See discussion in Section \ref{sec:ssl-verify-general}.  Defaults to True using the OS trusted CA certificates.

\subsubsection{YAML Element: timeout-seconds}\label{sec:yaml-generic-timeout-seconds}
The number of seconds before a request for API results times out.

\subsubsection{YAML Element: retries}\label{sec:yaml-generic-retries}
The number of retries when the request fails.


\subsubsection{YAML Element: proxies}\label{sec:yaml-generic-proxies}
A YAML dictionary of \texttt{<scheme>:<url>} pairs to use a proxy server for requests. 
For a format of key/value pairs, see: \extlink{https://requests.readthedocs.io/en/latest/user/advanced/\#proxies}{Python "requests" proxies}.










