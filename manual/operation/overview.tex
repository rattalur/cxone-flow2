\chapter{Overview}\label{sec:overview}

\section{Workflow Overview}

\cxoneflow has origins in CxFlow for the CxSAST product, which is the predecessor to Checkmarx One.  CxFlow
had a variety of functions and deployment options related to orchestrating scans in CxSAST and sending
result feedback to issue trackers.  \cxoneflow will also orchestrate scans but is adapted to the
concepts of CheckmarxOne.

The \cxoneflow logic for how scans are orchestrated is very similar to that of CxFlow.  The basic
logic flow is that scans are executed when:

\begin{itemize}
    \item If a Push is made to a repository's protected branch, that protected branch is scanned.
    \item If a Pull Request is opened that targets a protected branch, a scan is performed on
    the source branch.
    \item If a Push is made to a branch that is the source of an open Pull Request that targets
    a protected branch, that branch is scanned.
\end{itemize}


\cxoneflow follows this workflow logic upon the receipt of a webhook event payload generated by the SCM.
The code from the repository to be scanned is cloned by \cxoneflow, collected into a zip file, then submitted
for a scan to CheckmarxOne.  When the scan is submitted, the cloned code is deleted.


\section{Deployment Overview}

The method of deployment for \cxoneflow is intended to integrate scanning of all enterprise repositories
with a minimal amount of configuration.  The best method for deployment is to configure source control web
hooks where they will emit events for the largest possible number of repositories.  In many source control
systems, this can be done at a global or organization level.  The web hooks will be configured to send events
to a \cxoneflow endpoint specific to the type of source control system.

Figure \ref{fig:cxoneflow-deployment} is a \cxoneflow deployment diagram. The key points of the diagram:

\begin{itemize}
    \item A single instance or clustered install of \cxoneflow can be used as the endpoint for multiple
    SCM instances.
    \item There may be more than one instance of an SCM type.
    \item Each SCM instance may have one or more logical groups of repositories.  For example:
    \begin{itemize}
        \item Azure DevOps has \textbf{Collections} and each collection has a \textbf{Project} where
        each project will contain repositories.
        \item GitHub has \textbf{Organizations} where each organization will contain repositories.
        \item BitBucket Data Center has \textbf{Projects} where each project will contain repositories.
    \end{itemize}
    \item There may be multiple Checkmarx One tenants where scans are to be invoked from an SCM or SCM
    logical group of repositories.
\end{itemize}

The \cxoneflow configuration allows each endpoint to be configured such that it orchestrates
scans in the correct Checkmarx One instance and tenant for the SCM that emits the web hook event.
\cxoneflow is compatible with Checkmax One hosted single-tenant, hosted multi-tenant, and self-hosted
instances.

\begin{figure}[ht]
    \includegraphics[width=\textwidth]{graphics/cxoneflow-deployment.png}
    \caption{\cxoneflow Deployment Diagram}
    \label{fig:cxoneflow-deployment}
\end{figure}

