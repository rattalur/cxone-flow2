\subsubsection{public-key}\label{sec:agent-public-key}
The value specifies a file name found under the path defined by \texttt{secret-root-path} containing a 
public key that matches the server's configured \intlink{sec:yaml-resolver-private-key}{\texttt{private-key}} setting.

\subsubsection{resolver-opts}\label{sec:agent-resolver-opts}
This is a dictionary of
\extlink{https://docs.checkmarx.com/en/34965-132888-checkmarx-sca-resolver-configuration-arguments.html\#UUID-bc93274b-c1c7-ea47-9556-3bd8900711dc_id_CheckmarxSCAResolverConfigurationArguments-ConfigurationArguments-TablesandSamples}{configuration arguments}
passed to \scaresolver when executing.  The options are used to provide static values for resolver execution configuration.  Some of the
options may clash with execution options provided by the agent; options that would clash with how the agent executes \scaresolver are ignored.  

The \texttt{resolver-opts} section is a dictionary of key and key/value pairs. The \intlink{code:agent-yaml-example}{agent YAML configuration example}
shows the use of the following \scaresolver configuration parameters:

\begin{itemize}
  \item The key \texttt{log-level} with the value \texttt{Verbose} produces the configuration\\option: \texttt{--log-level=Verbose}
  \item The key \texttt{scan-containers} with no value produces the configuration\\option: \texttt{--scan-containers}
  \item The key \texttt{break-on-manifest-failure} with no value produces the configuration\\option: \texttt{--break-on-manifest-failure}
  \item The key \texttt{c} with the value \texttt{/etc/resolver/Configuration.yml} produces the configuration\\option: \texttt{-c=/etc/resolver/Configuration.yml}
\end{itemize}


These options, if set, will be ignored:

\begin{itemize}
  \item logs-path
  \item a | account
  \item containers-result-path
  \item resolver-result-path
  \item project-name
  \item authentication-server-url
  \item p | password
  \item sso-provider
  \item sca-app-url
  \item s | scan-path
  \item server-url
  \item u | username
  \item project-tags
  \item scan-tags
  \item bypass-exitcode
  \item no-upload-manifest
  \item help
  \item manifests-path
  \item t | project-teams
  \item q | quiet
  \item save-evidence-path
  \item severity-threshold
  \item report-content
  \item report-extension
  \item report-path
  \item report-type
  \item sast-result-path
  \item cxpassword
  \item cxuser
  \item cxprojectid
  \item cxprojectname
  \item cxserver
\end{itemize}

\subsubsection{resolver-path}\label{sec:agent-resolver-path}
The path to the \scaresolver executable that has been installed on the system running the distributed resolver agent.


\subsubsection{resolver-run-as}\label{sec:agent-resolver-run-as}
The name of a user account that will run the \scaresolver when executed in a shell (but not as a container).  This
is an advanced configuration that will require additional configuration for your platform.  

If not provided, the \scaresolver is executed as the same user that is running the distributed resolver agent service.

\subsubsection{resolver-work-path}\label{sec:agent-resolver-work-path}
A path where temporary files are written during the operation of the distributed resolver agent.  This also serves as the home
directory for the distributed resolver agent user and the user defined in \texttt{resolver-run-as}.

\subsubsection{run-with-container}\label{sec:agent-run-with-container}
A YAML dictionary with key/value pairs used to define running \scaresolver in a container.  If this is supplied,
the configured \texttt{resolver-path} is ignored and \scaresolver will not be invoked in a shell.  The dependency
tree collected by \scaresolver will be done by executing build tools defined in the container.  This is useful for
organizations that utilize containerized build environments in their CI/CD pipeline build scripts.

The use of containers to run \scaresolver is not supported on Windows platforms.

\subsubsection{container-image-tag}\label{sec:agent-container-image-tag}
The container tag that is found in one of the logged-in container registries.  This container tag is used by the \toolkit
to create an extended image with \scaresolver installed.

\subsubsection{supply-chain-toolkit-path}\label{sec:agent-supply-chain-toolkit-path}
The path where the \toolkit build environment is installed.

\subsubsection{use-running-gid and use-running-uid}\label{sec:agent-use-running}
These options are True by default.  This causes the image built by the \toolkit to use the
UID and primary GID of the user running the distributed resolver agent when defining a non-root
user in the extended image.  

The reason for this is that when \scaresolver is executed in the container, temporary paths
in the \texttt{resolver-work-path} are mapped to the container.  Files created by the container
will have the UID/GID of the running container's user when created.  Since the UID/GID of the
container matches the UID/GID of the distributed resolver agent, the files that remain after
the container exits can be controlled by the distributed resolver agent.

Setting these values to False should only be done in circumstances where the UID/GID for written files
should be defined by the container.  This scenario may never practically exist.

